% !TeX root = ../main.tex
\documentclass[
    ngerman,
    accentcolor=3b,
    dark_mode,
    fontsize= 12pt,
    a4paper,
    aspectratio=169,
    colorback=true,
    fancy_row_colors,
    leqno,
    fleqn,
    boxarc=3pt,
    fleqn,
    % shell_escape = false, % Kompatibilität mit sharelatex
]{algoslides}

%%--------------------------%%
%%--Imports from Main File--%%
%%--------------------------%%
\usepackage{import}
% Import all Packages from Main Preamble with relative Path (buggy, list packages instead)
\subimport*{../shared}{preamble}
% Get Labels from Main Document using the xr-hyper Package
\externaldocument[ext:]{../main}
% Set Graphics Path, so pictures load correctly
\graphicspath{{../pictures}}
\def\codeDir{../code}

\begin{document}
    \section{AlgoTeX}\label{AlgoTeX}
    \subsection{Tasks und Subtasks}
    \begin{frame}[fragile,c]
        \slidehead{}
        \vspace{-1.8em}
        \begin{codeBlock}[fontsize=\footnotesize]{minted language=latex, title=\codeBlockTitle{Beispielschema für Tasks und Subtasks},top=0cm,bottom=0cm}
        %--------------%
        %--Aufgabe H1--%
        %--------------%
        \begin{task}[points=<n>]{<Aufgabentitel>}\label{ex:H1} % Aufgabe ohne Subtasks
            <Aufgabenstellung>
        \end{task}
        %--------------%
        %--Aufgabe H2--%
        %--------------%
        \begin{task}[points=auto]{<Aufgabentitel>}\label{ex:H2} % Aufgabe mit Subtasks
            <Einleitung>
            % Aufgabe H2.1
            \begin{subtask*}[points=<n2>]{<Aufgabentitel>}\label{ex:H2.1}
                <Aufgabenstellung Teilaufgabe 1>
            \end{subtask*}
        \end{task}
    \end{codeBlock}
    \end{frame}
    \subsection{Boxen}
    \begin{frame}
        \slidehead{}
        \begin{columns}[c]
            \begin{column}[t]{\textwidth/3}
                \begin{defBox}
                    Das ist eine defBox
                \end{defBox}
                \begin{itemize}
                    \item sehr wichtige Informationen
                    \item z.B. verbindliche Anforderungen
                \end{itemize}
            \end{column}%
            \begin{column}[t]{\textwidth/3}
                \begin{infoBox}
                    Das ist eine infoBox
                \end{infoBox}
                \begin{itemize}
                    \item direkt die Übung betreffende informationen
                    \item nicht ganz so wichtig wie in der defBox
                    \item z.B. Hinweise/Beispiele
                \end{itemize}
            \end{column}%
            \begin{column}[t]{\textwidth/3}
                \begin{grayInfoBox}
                    Das ist eine grayInfoBox
                \end{grayInfoBox}
                \begin{itemize}
                    \item Informationen, die nicht relevant für die Übung sind
                    \item z.B. Hintergrundinformationen
                \end{itemize}
            \end{column}%
        \end{columns}
    \end{frame}
    \begin{frame}[c,fragile,allowframebreaks]
        % \begin{noindent}
        \begin{table}[ht!] \centering
            \rowcolors{2}{\thepagecolor}{fgcolor!10!\thepagecolor}
            \renewcommand{\arraystretch}{1}
            Übersicht vordefinierter Box-Environments
            \begin{longtable}{cccp{5cm}}
                \toprule
                \fatsf{Name} & \fatsf{Boxart} & \fatsf{default-Specifier}                                & \fatsf{Titel}                            \\
                \midrule
                \ExplSyntaxOn
                \seq_map_inline:Nn \rubos_color_box_environment_seq {
                    \bool_gset_true:N \rubos_color_box_environment_first_param_bool
                    \seq_map_inline:Nn {#1} {
                        \bool_if:NTF \rubos_color_box_environment_first_param_bool {
                            \bool_gset_false:N \rubos_color_box_environment_first_param_bool
                        } {
                            &
                        }
                        \tl_if_empty:nTF {##1} {\{\}} {##1}
                    }
                    \\
                }
                \ExplSyntaxOff%
                \\[-.5cm]\bottomrule
            \end{longtable}
        \end{table}
        % \end{noindent}
    \end{frame}
    \begin{frame}[c, fragile]
        \slidehead{}
        %\begin{noindent}
        \begin{codeBlock}[
            % highlightlines={7}
            ]{
            minted language=latex,
            listing and text,
            sidebyside,
            %righthand width=4.9cm,
            %center lower,
            bicolor,
            colbacklower=\IfDarkModeTF{codebg!90}{codebg!20},
            title=\codeBlockTitle{Beispiel für Boxen},
            }
            \begin{vanforderung}[Für H2.2]
                es darf keine Rekursion verwendet werden.
            \end{vanforderung}
        \end{codeBlock}
        %\end{noindent}
    \end{frame}
    \subsection{Code-Böcke}
    \let\newcb\codeBlock
    \let\endnewcb\endcodeBlock
    \begin{frame}[c, fragile]
        \slidehead{}
        %\begin{noindent}
        \begin{newcb}[
            % highlightlines={7}
            fontsize=\scriptsize,
            ]{
            minted language=latex,
            listing and text,
            sidebyside,
            %righthand width=4.9cm,
            %center lower,
            bicolor,
            colbacklower=\IfDarkModeTF{codebg!90}{codebg!20},
            title=\codeBlockTitle{Code-Block Beispiel},
            }
            \begin{codeBlock}[fontsize=\scriptsize]{
                minted language=java,
                title=\codeBlockTitle{Fibonacci},
            }
                public static int fibRec(int n) {
                    return n < 2 ? n : fibRec(n - 1) + fibRec(n - 2);
                }
            \end{codeBlock}
        \end{newcb}
        %\end{noindent}
    \end{frame}
    \begin{frame}[c, fragile]
        \slidehead{}
        \LaTeX-Code innerhalb eines Code-Blocks:
        %\begin{noindent}
        \begin{newcb}[
            % highlightlines={7}
            fontsize=\scriptsize,
            escapeinside=||,
            ]{
            minted language=latex,
            listing and text,
            sidebyside,
            %righthand width=4.9cm,
            %center lower,
            bicolor,
            colbacklower=\IfDarkModeTF{codebg!90}{codebg!20},
            title=\codeBlockTitle{Code-Block Beispiel2},
            }
            \begin{codeBlock}[
                fontsize=\scriptsize,
                escapeinside=@@
            ]{
                minted language=java,
                title=\codeBlockTitle{SomeFormula},
            }
                return @\alpha@ + @\beta@;
            \end{codeBlock}
        \end{newcb}
        %\end{noindent}
    \end{frame}
    \begin{frame}[c, fragile]
        \slidehead{}
        Auch Inline möglich:
        %\begin{noindent}
        \begin{newcb}[
            % highlightlines={7}
            fontsize=\scriptsize,
            escapeinside=||,
            ]{
            minted language=latex,
            listing and text,
            sidebyside,
            %righthand width=4.9cm,
            %center lower,
            bicolor,
            colbacklower=\IfDarkModeTF{codebg!90}{codebg!20},
            title=\codeBlockTitle{Code-Block Beispiel2},
            }
            Schreiben Sie eine \inlinejava{public}-Klassenmethode \inlinejava{fibRec(int n)}, die die $n$-te Fibonacci-Zahl rekursiv berechnet.
        \end{newcb}
        %\end{noindent}
    \end{frame}
    \subsection{Javadoc + Referenzen}
    \begin{frame}[c, fragile]
        \slidehead{}
        Referenzen
        %\begin{noindent}
        \begin{newcb}[
            % highlightlines={7}
            fontsize=\scriptsize,
            escapeinside=||,
            ]{
            minted language=latex,
            listing and text,
            sidebyside,
            %righthand width=4.9cm,
            %center lower,
            bicolor,
            colbacklower=\IfDarkModeTF{codebg!90}{codebg!20},
            title=\codeBlockTitle{Code-Block Beispiel2},
            }
            \label{lst:java:example}
            \begin{codeBlock}[fontsize=\scriptsize]{
            }
                System.out.println("Hello World");
            \end{codeBlock}

            Wie in \ref{lst:java:example} zu sehen, wird \inlinejava{\"Hello World\"} ausgegeben.
        \end{newcb}
        %\end{noindent}
    \end{frame}
    \begin{frame}[c, fragile]
        \slidehead{}
        Referenzen (besser)
        %\begin{noindent}
        \begin{newcb}[
            % highlightlines={7}
            fontsize=\scriptsize,
            escapeinside=||,
            ]{
            minted language=latex,
            listing and text,
            sidebyside,
            %righthand width=4.9cm,
            %center lower,
            bicolor,
            colbacklower=\IfDarkModeTF{codebg!90}{codebg!20},
            title=\codeBlockTitle{Code-Block Beispiel2},
            }
            \label{fig:example}
            \begin{codeBlock}[fontsize=\scriptsize]{
            }
                System.out.println("Hello World");
            \end{codeBlock}

            Wie in \figurename{}~\ref{fig:example} zu sehen, wird \inlinejava{\"Hello World\"} ausgegeben.
        \end{newcb}
        %\end{noindent}
    \end{frame}
    \begin{frame}[c, fragile]
        \slidehead{}
        Referenzen (gut)
        %\begin{noindent}
        \begin{newcb}[
            % highlightlines={7}
            fontsize=\scriptsize,
            escapeinside=||,
            ]{
            minted language=latex,
            listing and text,
            sidebyside,
            %righthand width=4.9cm,
            %center lower,
            bicolor,
            colbacklower=\IfDarkModeTF{codebg!90}{codebg!20},
            title=\codeBlockTitle{Code-Block Beispiel2},
            }
            \label{fig:example}
            \begin{codeBlock}[fontsize=\scriptsize]{
            }
                System.out.println("Hello World");
            \end{codeBlock}

            Wie im \hyperref[fig:example]{Code-Beispiel} zu sehen, wird \inlinejava{\"Hello World\"} ausgegeben.
        \end{newcb}
        %\end{noindent}
    \end{frame}
    \begin{frame}[c, fragile]
        \slidehead{}
        \verb+\refJavaDoc{}+:
        %\begin{noindent}
        \begin{newcb}[
            % highlightlines={7}
            fontsize=\scriptsize,
            escapeinside=||,
            ]{
            minted language=latex,
            listing and text,
            sidebyside,
            %righthand width=4.9cm,
            %center lower,
            bicolor,
            colbacklower=\IfDarkModeTF{codebg!90}{codebg!20},
            title=\codeBlockTitle{Code-Block Beispiel2},
            }
            \begin{grayInfoBox}
                \refJavaDoc{java.util.ArrayList}
            \end{grayInfoBox}
        \end{newcb}
        %\end{noindent}
    \end{frame}
    \begin{frame}[c, fragile]
        \slidehead{}
        \verb+\refJavaDoc{}+ spezifische Methode:
        %\begin{noindent}
        \begin{newcb}[
            % highlightlines={7}
            fontsize=\scriptsize,
            escapeinside=||,
            ]{
            minted language=latex,
            listing and text,
            sidebyside,
            %righthand width=4.9cm,
            %center lower,
            bicolor,
            colbacklower=\IfDarkModeTF{codebg!90}{codebg!20},
            title=\codeBlockTitle{Code-Block Beispiel2},
            }
            \begin{grayInfoBox}
                \refJavaDoc[jdk=11, ref={add(int,E)}]{java.util.ArrayList}
            \end{grayInfoBox}
        \end{newcb}
        %\end{noindent}
    \end{frame}
    \begin{frame}[c, fragile]
        \slidehead{}
        \verb+\refJavaDoc{}+ spezifische Methode:
        %\begin{noindent}
        \begin{newcb}[
            % highlightlines={7}
            fontsize=\scriptsize,
            escapeinside=||,
            ]{
            minted language=latex,
            listing and text,
            sidebyside,
            %righthand width=4.9cm,
            %center lower,
            bicolor,
            colbacklower=\IfDarkModeTF{codebg!90}{codebg!20},
            title=\codeBlockTitle{Code-Block Beispiel2},
            }
            \begin{grayInfoBox}
                \refJavaDoc[jdk=17, title=Methode \inlinejava{size} von \inlinejava{ArrayList}, ref=size()]{java.util.ArrayList}
            \end{grayInfoBox}
        \end{newcb}
        %\end{noindent}
    \end{frame}
    \begin{frame}[c, fragile]
        \slidehead{}
        \verb+\refJavaDoc{}+ und \verb+\refJavaDoc*{}+:
        %\begin{noindent}
        \begin{newcb}[
            % highlightlines={7}
            fontsize=\scriptsize,
            escapeinside=||,
            ]{
            minted language=latex,
            listing and text,
            sidebyside,
            %righthand width=4.9cm,
            %center lower,
            bicolor,
            colbacklower=\IfDarkModeTF{codebg!90}{codebg!20},
            title=\codeBlockTitle{Code-Block Beispiel2},
            }
            \begin{grayInfoBox}
                \textbf{\emph{Fehlermeldungen besser verstehen}:}
                Was passiert, wenn Sie den generischen Typparameter \inlinejava{T} bei einer dieser Klassen versuchsweise mit
                \refJavaDoc*{java.lang.String} oder \refJavaDoc*{java.lang.Object} instantiieren?
            \end{grayInfoBox}
        \end{newcb}
        %\end{noindent}
    \end{frame}
    \subsection{Übungsphase}
    \begin{frame}[c]
        \slidehead{}
        \centering
        5 Minuten Übungsphase
        \begin{columns}[c]
            \begin{column}{.5\textwidth}
                \begin{figure}
                    \centering
                    \qrcode[height=3cm]{https://sharelatex.tu-darmstadt.de}
                    \caption{Sharelatex-Instanz\\\url{https://sharelatex.tu-darmstadt.de}}
                \end{figure}
            \end{column}%
            \begin{column}{.5\textwidth}
                \begin{figure}
                    \centering
                    \qrcode[height=3cm]{https://github.com/Rdeisenroth/LaTeX-Workshop/releases/latest}
                    \caption{Link zu Folien + Übungen\\\url{https://github.com/Rdeisenroth/LaTeX-Workshop/releases/latest}}
                \end{figure}
            \end{column}
        \end{columns}
    \end{frame}
\end{document}
