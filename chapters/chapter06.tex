% !TeX root = ../main.tex
\documentclass[
    ngerman,
    accentcolor=3b,
    dark_mode,
    fontsize= 12pt,
    a4paper,
    aspectratio=169,
    colorback=true,
    fancy_row_colors,
    leqno,
    fleqn,
    boxarc=3pt,
    fleqn,
    % shell_escape = false, % Kompatibilität mit sharelatex
]{algoslides}

%%--------------------------%%
%%--Imports from Main File--%%
%%--------------------------%%
\usepackage{import}
% Import all Packages from Main Preamble with relative Path (buggy, list packages instead)
\subimport*{../shared}{preamble}
% Get Labels from Main Document using the xr-hyper Package
\externaldocument[ext:]{../main}
% Set Graphics Path, so pictures load correctly
\graphicspath{{../pictures}}
\def\codeDir{../code}

\begin{document}
    \section{Challenges}\label{4}\label{Challenges}
    \subsection{Aufgabe 1: Hello World}
    \begin{frame}[fragile]
        \slidehead{}
        Minimales Textdokument mit Klasse \texttt{article} und \texttt{a4paper} mit dem Text \verb+Hello World+.
    \end{frame}
    \subsection{Aufgabe 2: Blindtext}
    \begin{frame}[fragile]
        \slidehead{}
        Textdokument mit 2 seiten Blindtext
    \end{frame}
    \subsection{Aufgabe 3: Mathematische Formel}
    \begin{frame}[fragile]
        \slidehead{}
        Tippen Sie die folgende Formel ein:
        \begin{defBox}
            \fatsf{Mitternachtsformel}
            Für eine Polynom zweiten Grades in der Form $a\cdot x^2+b\cdot x + c$ \\gilt für $a,b,c \in \mathbb{R}$ immer:%

            \begin{align}
                x_{1/2} & =\frac{-b\pm\sqrt{b^{2}-4\cdot a \cdot c}}{2\cdot a}
            \end{align}

        \end{defBox}
    \end{frame}
    \subsection{Aufgabe 4: Tabelle}
    \begin{frame}[fragile]
        \slidehead{}
        Tippen Sie die folgende Formel ein:
        \begin{table}[ht]
            \centering
            \begin{tabular}{l|c|r}
                \textbf{links} & \textbf{mitte} & \textbf{rechts} \\
                \hline
                a              & b              & c               \\
                d              & e              & f               \\
                g              & h              & i               \\
                j              & k              & l
            \end{tabular}
            \caption{Test-Tabelle}
            \label{tab:Test}
        \end{table}
    \end{frame}
\end{document}
