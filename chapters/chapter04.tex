% !TeX root = ../main.tex
\documentclass[
    ngerman,
    accentcolor=3b,
    dark_mode,
    fontsize= 12pt,
    a4paper,
    aspectratio=169,
    colorback=true,
    fancy_row_colors,
    leqno,
    fleqn,
    boxarc=3pt,
    fleqn,
    % shell_escape = false, % Kompatibilität mit sharelatex
]{algoslides}

%%--------------------------%%
%%--Imports from Main File--%%
%%--------------------------%%
\usepackage{import}
% Import all Packages from Main Preamble with relative Path (buggy, list packages instead)
\subimport*{../shared}{preamble}
% Get Labels from Main Document using the xr-hyper Package
\externaldocument[ext:]{../main}
% Set Graphics Path, so pictures load correctly
\graphicspath{{../pictures}}
\def\codeDir{../code}

\begin{document}
    \section{Mathematische Formeln}
    \subsection{Unterschiedliche Mathemodi}
    \begin{frame}[c, fragile]
        \slidehead{}
        %\begin{noindent}
        \begin{codeBlock}[]{
            minted language=latex,
            listing and text,
            sidebyside,
            %righthand width=3.5cm,
            center lower,
            bicolor,
            colbacklower=\IfDarkModeTF{codebg!90}{codebg!20},
            title=\codeBlockTitle{Inline-Mathe Modus}
            }
            die Formel $a^2+b^2=c^2$ ist der Pythagoras-Satz.
        \end{codeBlock}
        %\end{noindent}
        \begin{itemize}
            \item Inline-Mathe Modus ist für kurze Formeln geeignet
            \item Die Formeln werden im Fließtext platziert
        \end{itemize}
    \end{frame}

    \begin{frame}[c, fragile]
        \slidehead{}
        %\begin{noindent}
        \begin{codeBlock}[]{
            minted language=latex,
            listing and text,
            sidebyside,
            righthand width=4.5cm,
            center lower,
            bicolor,
            colbacklower=\IfDarkModeTF{codebg!90}{codebg!20},
            title=\codeBlockTitle{Display-Mathe Modus}
            }
            die Formel $$a^2+b^2=c^2$$ ist der Pythagoras-Satz.
        \end{codeBlock}
        %\end{noindent}
        \begin{itemize}
            \item Display-Mathe Modus ist für längere Formeln geeignet
            \item Formeln werden in eigener Zeile zentriert angezeigt
        \end{itemize}
    \end{frame}

    \subsection{Mathe-Modus vs Text-Modus}
    \begin{frame}[c,fragile]
        \slidehead{}
        Im Mathe-Modus:
        \begin{itemize}
            \item werden Leerzeichen ignoriert
            \item werden Zeilenumbrüche ignoriert
            \item haben die meisten Zeichen eine besondere Bedeutung (z.B. \verb|_| und \verb|^|)
            \item können mathematische Symbole verwendet werden
            \item können keine \enquote{reinen Text-Befehle} verwendet werden
        \end{itemize}
    \end{frame}

    \begin{frame}[c, fragile]
        \slidehead{}
        %\begin{noindent}
        \begin{codeBlock}[]{
            minted language=latex,
            listing and text,
            sidebyside,
            righthand width=4.5cm,
            center lower,
            bicolor,
            colbacklower=\IfDarkModeTF{codebg!90}{codebg!20},
            title=\codeBlockTitle{Textarten im Text-Modus}
            }
            \textbf{Fetter Text}\\
            \textit{Kursiver Text}\\
            \texttt{Monospaced Text}\\
            \textsc{Kapitälchen}\\
            \textsf{Sans Serif}\\
            \textsl{Schräg}\\
            \textup{Normal}\\
            \textmd{Medium}\\
            \textbf{\textit{Fetter Kursiver Text}}
        \end{codeBlock}
        %\end{noindent}
    \end{frame}

    \begin{frame}[c, fragile]
        \slidehead{}
        %\begin{noindent}
        \begin{codeBlock}[]{
            minted language=latex,
            listing and text,
            sidebyside,
            righthand width=4cm,
            center lower,
            bicolor,
            colbacklower=\IfDarkModeTF{codebg!90}{codebg!20},
            title=\codeBlockTitle{Textarten im Mathe-Modus}
            }
            $\mathbf{Fetter Text}$\\
            $\mathit{Kursiver Text}$\\
            $\mathtt{Monospaced Text}$\\
            $\mathsf{Sans Serif}$\\
            $\mathrm{Normal}$\\
            $\mathbf{\mathit{Fetter Kursiver Text}}$
            $\text{Text im Mathe-Modus}$
        \end{codeBlock}
        %\end{noindent}
    \end{frame}

    \begin{frame}[c, fragile]
        \slidehead{}
        %\begin{noindent}
        \begin{codeBlock}[highlightlines={7}]{
            minted language=latex,
            listing and text,
            sidebyside,
            righthand width=4cm,
            center lower,
            bicolor,
            colbacklower=\IfDarkModeTF{codebg!90}{codebg!20},
            title=\codeBlockTitle{Textarten im Mathe-Modus}
            }
            $\mathbf{Fetter Text}$\\
            $\mathit{Kursiver Text}$\\
            $\mathtt{Monospaced Text}$\\
            $\mathsf{Sans Serif}$\\
            $\mathrm{Normal}$\\
            $\mathbf{\mathit{Fetter Kursiver Text}}$
            $\text{Text im Mathe-Modus}$
        \end{codeBlock}
        %\end{noindent}
    \end{frame}

    \subsection{Wichtige Mathematik-Operatoren}
    \begin{frame}[c, fragile]
        \slidehead{}
        %\begin{noindent}
        \begin{codeBlock}[
            % highlightlines={7}
            ]{
            minted language=latex,
            listing and text,
            sidebyside,
            righthand width=4.9cm,
            %center lower,
            bicolor,
            colbacklower=\IfDarkModeTF{codebg!90}{codebg!20},
            title=\codeBlockTitle{Hoch- und Tiefstellen}
            }
            $$a^2$$ % Hochstellen
            $$a_2$$ % Tiefstellen
            $$a^2_2$$ % Hoch- und Tiefstellen
            $$a_2^2bc$$
            $$a^{2n}b$$
            $$a_{2n}b$$
        \end{codeBlock}
        %\end{noindent}
    \end{frame}

    \begin{frame}[c, fragile]
        \slidehead{}
        %\begin{noindent}
        \begin{codeBlock}[
            % highlightlines={7}
            ]{
            minted language=latex,
            listing and text,
            sidebyside,
            righthand width=4.9cm,
            %center lower,
            bicolor,
            colbacklower=\IfDarkModeTF{codebg!90}{codebg!20},
            title=\codeBlockTitle{Grichische Buchstaben}
            }
            $$\alpha$$
            $$\beta$$
            $$\gamma$$
            $$\Gamma$$
            $$\dots$$
        \end{codeBlock}
        %\end{noindent}
    \end{frame}

    \begin{frame}[c, fragile]
        \slidehead{}
        %\begin{noindent}
        \begin{codeBlock}[
            % highlightlines={7}
            ]{
            minted language=latex,
            listing and text,
            sidebyside,
            righthand width=4.9cm,
            %center lower,
            bicolor,
            colbacklower=\IfDarkModeTF{codebg!90}{codebg!20},
            title=\codeBlockTitle{Modifikatoren}
            }
            $$\underline{\phi}$$
            $$\overline{\varphi}$$
            $$\overbrace{(a+b)}^{\text{Teil a}}$$
            $$\underbrace{(a+b)}_{\text{Teil b}}$$
        \end{codeBlock}
        %\end{noindent}
    \end{frame}

    \begin{frame}[c, fragile]
        \slidehead{}
        %\begin{noindent}
        \begin{codeBlock}[
            % highlightlines={7}
            ]{
            minted language=latex,
            listing and text,
            sidebyside,
            righthand width=4.9cm,
            %center lower,
            bicolor,
            colbacklower=\IfDarkModeTF{codebg!90}{codebg!20},
            title=\codeBlockTitle{Modifikatoren (2)}
            }
            $$\hat{\delta}$$
            $$\vec{a}$$
            $$\dot{\psi}$$
            $$\ddot{\Psi}$$
        \end{codeBlock}
        %\end{noindent}
    \end{frame}

    \begin{frame}[c, fragile]
        \slidehead{}
        %\begin{noindent}
        \begin{codeBlock}[
            % highlightlines={7}
            ]{
            minted language=latex,
            listing and text,
            sidebyside,
            righthand width=4.9cm,
            %center lower,
            bicolor,
            colbacklower=\IfDarkModeTF{codebg!90}{codebg!20},
            title=\codeBlockTitle{Pfeile}
            }
            $$\leftarrow, \rightarrow$$
            $$\uparrow, \downarrow$$
            $$\leftrightarrow, \updownarrow$$
            $$\Leftarrow$$
            $$\longleftarrow$$
        \end{codeBlock}
        %\end{noindent}
    \end{frame}

    \begin{frame}[c, fragile]
        \slidehead{}
        %\begin{noindent}
        \begin{codeBlock}[
            % highlightlines={7}
            ]{
            minted language=latex,
            listing and text,
            sidebyside,
            righthand width=4.9cm,
            %center lower,
            bicolor,
            colbacklower=\IfDarkModeTF{codebg!90}{codebg!20},
            title=\codeBlockTitle{Relationen}
            }
            $$<, >, \leq, \geq$$
            $$=, \neq, \approx$$
            $$\equiv, \sim$$
            $$\in, \notin$$
            $$\subset, \supset$$
            $$\subseteq, \supseteq$$
        \end{codeBlock}
        %\end{noindent}
    \end{frame}

    \begin{frame}[fragile]
        \slidehead{}
        Und noch viel mehr...

        \vspace{\fill}
        \begin{center}
            \Large\fatsf{Wie soll man sich da bitte noch zurechtfinden???}
        \end{center}
    \end{frame}

    \urlslide[Detexify - Symbol Zeichnen und Befehl finden]{https://detexify.kirelabs.org}

    \begin{frame}[c, fragile]
        \slidehead{}
        %\begin{noindent}
        \begin{codeBlock}[
            % highlightlines={7}
            ]{
            minted language=latex,
            listing and text,
            sidebyside,
            righthand width=4.9cm,
            %center lower,
            bicolor,
            colbacklower=\IfDarkModeTF{codebg!90}{codebg!20},
            title=\codeBlockTitle{Brüche}
            }
            $$\frac{a}{2\cdot b}$$ % Bruch
            $$\frac{a}{\frac{b}{c}}$$ % Bruch in Bruch
        \end{codeBlock}
        %\end{noindent}
    \end{frame}

    \begin{frame}[c, fragile]
        \slidehead{}
        %\begin{noindent}
        \begin{codeBlock}[
            % highlightlines={7}
            ]{
            minted language=latex,
            listing and text,
            sidebyside,
            righthand width=3cm,
            %center lower,
            bicolor,
            colbacklower=\IfDarkModeTF{codebg!90}{codebg!20},
            title=\codeBlockTitle{Große Klammern}
            }
            $$2\cdot(\frac{a}{b}+\frac{c}{d})$$
            $$2\cdot\left(\frac{a}{b}+\frac{c}{d}\right)$$
            $$F=\left.
                \frac{\partial f}{\partial x}
                \right|_{\hat x_{k-1}}$$
        \end{codeBlock}
        %\end{noindent}
    \end{frame}

    \begin{frame}[fragile]
        \slidehead{}
        \vspace{\fill}
        \begin{center}
            \Large\fatsf{Uff, das ist aber immer noch recht kompliziert, gibt's da nicht noch eine andere Möglichkeit?!}
        \end{center}
    \end{frame}

    \urlslide[LaTeX-OCR - OCR-Erkennung von LaTeX-Formeln]{https://github.com/lukas-blecher/LaTeX-OCR}

    \subsection{Übungsphase}
    \begin{frame}[c]
        \slidehead{}
        \centering
        5 Minuten Übungsphase
        \begin{columns}[c]
            \begin{column}{.5\textwidth}
                \begin{figure}
                    \centering
                    \qrcode[height=3cm]{https://coder-ophase.ruben-deisenroth.de}
                    \caption{Coder-Instanz\\\url{https://coder-ophase.ruben-deisenroth.de}}
                \end{figure}
            \end{column}%
            \begin{column}{.5\textwidth}
                \begin{figure}
                    \centering
                    \qrcode[height=3cm]{https://github.com/Rdeisenroth/LaTeX-Workshop/releases/latest}
                    \caption{Link zu Folien + Übungen\\\url{https://github.com/Rdeisenroth/LaTeX-Workshop/releases/latest}}
                \end{figure}
            \end{column}
        \end{columns}
    \end{frame}
\end{document}
