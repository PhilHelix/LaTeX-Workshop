% !TeX root = ../main.tex
\documentclass[
    ngerman,
    accentcolor=3b,
    dark_mode,
    fontsize= 12pt,
    a4paper,
    aspectratio=169,
    colorback=true,
    fancy_row_colors,
    leqno,
    fleqn,
    boxarc=3pt,
    fleqn,
    % shell_escape = false, % Kompatibilität mit sharelatex
]{algoslides}

%%--------------------------%%
%%--Imports from Main File--%%
%%--------------------------%%
\usepackage{import}
% Import all Packages from Main Preamble with relative Path (buggy, list packages instead)
\subimport*{../shared}{preamble}
% Get Labels from Main Document using the xr-hyper Package
\externaldocument[ext:]{../main}
% Set Graphics Path, so pictures load correctly
\graphicspath{{../pictures}}
\def\codeDir{../code}

\begin{document}
    \section{\LaTeX-Basics}\label{2}\label{LaTeX-Basics}
    \subsection{Syntax}
    \begin{frame}[fragile]
        \slidehead{}
        \begin{itemize}
            \item Zeichen mit besonderer Bedeutung: \verb+\+, \verb+&+, \verb+#+, \verb+$+, \verb+%+, \verb+~+, \verb+^+, \verb+_+, \verb+{+, \verb+}+, \verb+[+ und \verb+]+
            \item Kommandos beginnen mit einem Backslask: \verb+\+
            \item Kommandonamen enthalten keine Zahlen, Leerzeichen und Sonderzeichen
            \item Argumente werden in geschweiften Klammern angegeben \verb+{}+
            \item Optionale Argumente werden in eckigen Klammern angegeben \verb+[]+
            \item Environments werden mit \verb+\begin{<name>}+ und \verb+\end{<name>}+ angegeben
            \item Kommentare beginnen mit einem \verb+%+
            \item Parameter werden mit \verb+#+ angegeben und gehen von 1-9\begin{itemize}
                    \item Dabei wird für innere Argumente die Anzahl der \verb+#+ verdoppelt (z.B. \verb+#1+, \verb+##1+, \verb+####1+)
                \end{itemize}
        \end{itemize}
    \end{frame}
    \subsection{Aufbau eines Dokumentes}\label{2.1}\label{2.1}
    \begin{frame}[fragile]
        \slidehead{}
        \begin{columns}[c]
            \begin{column}{.5\textwidth}
                \inputCode[]{
                    minted language=latex,
                    title=\codeBlockTitle{Minimales \LaTeX-Dokument},
                }{\codeDir/minimal.tex}
            \end{column}%
            \begin{column}{.5\textwidth}
                \centering
                \colorbox{white}{%
                    \color{black}
                    \includegraphics[width=\textwidth,height=5cm,keepaspectratio]{\codeDir/minimal}
                }
            \end{column}
        \end{columns}
    \end{frame}
    \subsubsection{Dokumentenklasse}
    \begin{frame}[fragile]
        \slidehead{}
        \begin{columns}[c]
            \begin{column}{.5\textwidth}
                \inputCode[highlightlines={1}]{
                    minted language=latex,
                    title=\codeBlockTitle{Minimales \LaTeX-Dokument},
                }{\codeDir/minimal.tex}
            \end{column}%
            \begin{column}{.5\textwidth}
                \centering
                \colorbox{white}{%
                    \color{black}
                    \includegraphics[width=\textwidth,height=5cm,keepaspectratio]{\codeDir/minimal}
                }
            \end{column}
        \end{columns}
    \end{frame}

    \begin{frame}[c,fragile]
        \slidehead{}
        Häufig verwendete Dokumentenklassen:
        \begin{itemize}
            \item \verb+article+ (Standardklasse für allgemeine Dokumente)
            \item \verb+book+ (Bücher)
            \item \verb+beamer+ (Präsentationen)
            \item \verb+standalone+ (Allenstehende Abbildungen)
        \end{itemize}
        \pause{}
        Klassen des Corporatedesigns der TU Darmstadt:
        \begin{itemize}
            \item \verb+tudapub+ (Standardklasse für allgemeine Publikationen)
            \item \verb+tudaexercise+ (Übungsblätter)
            \item \verb+tudabeamer+ (Präsentationen)
        \end{itemize}

        \pause{}Und viele mehr...
    \end{frame}

    \subsubsection{Dokumenteninhalt}
    \begin{frame}[fragile]
        \slidehead{}
        \begin{columns}[c]
            \begin{column}{.5\textwidth}
                \inputCode[highlightlines={2,4}]{
                    minted language=latex,
                    title=\codeBlockTitle{Minimales \LaTeX-Dokument},
                }{\codeDir/minimal.tex}
            \end{column}%
            \begin{column}{.5\textwidth}
                \centering
                \colorbox{white}{%
                    \color{black}
                    \includegraphics[width=\textwidth,height=5cm,keepaspectratio]{\codeDir/minimal}
                }
            \end{column}
        \end{columns}
    \end{frame}

    \begin{frame}[fragile]
        \slidehead{}
        \begin{columns}[c]
            \begin{column}{.5\textwidth}
                \inputCode[highlightlines={3}]{
                    minted language=latex,
                    title=\codeBlockTitle{Minimales \LaTeX-Dokument},
                }{\codeDir/minimal.tex}
            \end{column}%
            \begin{column}{.5\textwidth}
                \centering
                \colorbox{white}{%
                    \color{black}
                    \includegraphics[width=\textwidth,height=5cm,keepaspectratio]{\codeDir/minimal}
                }
            \end{column}
        \end{columns}
    \end{frame}

    \subsubsection{Pakete}

    \begin{frame}[fragile]
        \slidehead{}
        \begin{columns}[c]
            \begin{column}{.5\textwidth}
                \inputCode[highlightlines={3}]{minted language=latex,title=\codeBlockTitle{Minimales \LaTeX-Dokument}}{\codeDir/minimal-lipsum.tex}
            \end{column}%
            \begin{column}{.5\textwidth}
                \centering
                \colorbox{white}{%
                    \color{black}
                    \includegraphics[width=\textwidth,height=5cm,keepaspectratio]{\codeDir/minimal-lipsum}
                }
            \end{column}
        \end{columns}
    \end{frame}

    \begin{frame}[fragile]
        \slidehead{}
        \begin{columns}[c]
            \begin{column}{.5\textwidth}
                \inputCode[highlightlines={5}]{minted language=latex,title=\codeBlockTitle{Minimales \LaTeX-Dokument}}{\codeDir/minimal-lipsum.tex}
            \end{column}%
            \begin{column}{.5\textwidth}
                \centering
                \colorbox{white}{%
                    \color{black}
                    \includegraphics[width=\textwidth,height=5cm,keepaspectratio]{\codeDir/minimal-lipsum}
                }
            \end{column}
        \end{columns}
    \end{frame}

    \subsection{Kompilieren}
    \begin{frame}[c, fragile]
        \slidehead{}
        \begin{figure}
            \centering
            % three nodes: source -> compile -> output
            \tikzset{
                graphnode/.style={draw, rectangle, rounded corners, minimum width=2cm, minimum height=1cm, text width=2cm, align=center, fill=#1,font=\color{white}},
            }
            \begin{tikzpicture}[start chain=1 going right, every on chain/.style={join=by -Latex}, thick]
                \node<1->[graphnode=TUDa-1b, on chain=1] (source) {Quelldatei};
                \node<3->[graphnode=TUDa-8b, on chain=1] (compile) {Kompilieren};
                \node<5->[graphnode=TUDa-3b, on chain=1] (output) {Ausgabe};

                \node<2->[above=0cm of source]{*.tex};
                \node<4->[above=0cm of compile]{latexmk <datei.tex>};
                \node<6->[above=0cm of output]{*.pdf/*.dvi/*.ps};
            \end{tikzpicture}
            \caption{Kompilieren mit \LaTeX}
        \end{figure}
        \begin{itemize}
            \item<7-> In der Praxis klickt man meistens auf einen Button in der IDE
            \item<8-> \verb+latexmk+ ist ein Wrapper, der die Kompilierung automatisiert
        \end{itemize}
    \end{frame}

    \subsection{Arbeiten mit Text}
    \begin{frame}[c, fragile]
        \slidehead{}
        %\begin{noindent}
        \begin{codeBlock}[]{
            minted language=latex,
            listing and text,
            sidebyside,
            righthand width=4.5cm,
            %center lower,
            fontlower=\setlength{\parskip}{\medskipamount}\setlength{\parindent}{2em},
            bicolor,
            colbacklower=\IfDarkModeTF{codebg!90}{codebg!20},
            title=\codeBlockTitle{Zeilenumbrüche}
            }
            Das ist ein Zeilen-\\umbruch. Er beginnt eine neue Zeile.

            Das ist ein Zeilen-\linebreak umbruch. Er beginnt eine neue Zeile.

            Und hier beginnt ein neuer Absatz.
        \end{codeBlock}
        %\end{noindent}
    \end{frame}

    \begin{frame}[c, fragile]
        \slidehead{}
        %\begin{noindent}
        \begin{codeBlock}[]{
            minted language=latex,
            listing and text,
            sidebyside,
            righthand width=4.5cm,
            %center lower,
            fontlower=\setlength{\parskip}{\medskipamount}\setlength{\parindent}{2em},
            bicolor,
            colbacklower=\IfDarkModeTF{codebg!90}{codebg!20},
            title=\codeBlockTitle{Zeilenumbrüche}
            }
            \setlength{\parindent}{0em}
            Das ist ein Zeilen-\\umbruch. Er beginnt eine neue Zeile.

            Das ist ein Zeilen-\linebreak umbruch. Er beginnt eine neue Zeile.

            Und hier beginnt ein neuer Absatz.
        \end{codeBlock}
        %\end{noindent}
    \end{frame}

    \begin{frame}[c, fragile]
        \slidehead{}
        Seitenumbrüche:
        \begin{itemize}
            \item \verb+\clearpage+ oder \verb+\cleardoublepage+
            \item \verb+\newpage+ oder \verb+\newdoublepage+
        \end{itemize}
    \end{frame}

    \begin{frame}[c, fragile]
        \slidehead{}
        %\begin{noindent}
        \begin{codeBlock}[]{
            minted language=latex,
            listing and text,
            sidebyside,
            righthand width=4.5cm,
            %center lower,
            fontlower=\setlength{\parskip}{\medskipamount}\setlength{\parindent}{0em},
            bicolor,
            colbacklower=\IfDarkModeTF{codebg!90}{codebg!20},
            title=\codeBlockTitle{Trennhilfen}
            }
            Das längste Deutsche Wort lautet nunmal Rindfleischetikettierungs%
            überwachungsaufgabenübertragungsgesetz.

            Das längste Deutsche Wort lautet nunmal Rindfleisch\-etikettierungs%
            überwachungs\-auf\-gaben%
            \-übertragungs\-gesetz.
        \end{codeBlock}
        %\end{noindent}
    \end{frame}

    \begin{frame}[c, fragile]
        \slidehead{}
        %\begin{noindent}
        \begin{codeBlock}[]{
            minted language=latex,
            listing and text,
            sidebyside,
            righthand width=4.5cm,
            %center lower,
            fontlower=\setlength{\parskip}{\medskipamount}\setlength{\parindent}{0em},
            bicolor,
            colbacklower=\IfDarkModeTF{codebg!90}{codebg!20},
            title=\codeBlockTitle{Geschützte Leerzeichen}
            }
            Ich möchte dass diese
            \enquote{zwei Wörter}
            nicht getrennt werden.

            Ich möchte dass diese
            \enquote{zwei~Wörter}
            nicht getrennt werden.
        \end{codeBlock}
        %\end{noindent}
    \end{frame}

    \begin{frame}[c, fragile]
        \slidehead{}
        %\begin{noindent}
        \begin{codeBlock}[]{
            minted language=latex,
            listing and text,
            sidebyside,
            righthand width=4.5cm,
            center lower,
            bicolor,
            colbacklower=\IfDarkModeTF{codebg!90}{codebg!20},
            title=\codeBlockTitle{Textarten}
            }
            \textbf{Fetter Text}\\
            \textit{Kursiver Text}\\
            \texttt{Monospaced Text}\\
            \textsc{Kapitälchen}\\
            \textsf{Sans Serif}\\
            \textsl{Schräg}\\
            \textup{Normal}\\
            \textmd{Medium}\\
            \textbf{\textit{Fetter Kursiver Text}}
        \end{codeBlock}
        %\end{noindent}
    \end{frame}

    \begin{frame}[c, fragile]
        %\slidehead{}
        %\begin{noindent}
        \begin{codeBlock}[]{
            minted language=latex,
            listing and text,
            sidebyside,
            righthand width=4.5cm,
            center lower,
            bicolor,
            colbacklower=\IfDarkModeTF{codebg!90}{codebg!20},
            title=\codeBlockTitle{Schriftgrößen}
            }
            \tiny{Klein}\\
            \scriptsize{Klein}\\
            \footnotesize{Klein}\\
            \small{Klein}\\
            \normalsize{Normal}\\
            \large{Groß}\\
            \Large{Groß}\\
            \LARGE{Groß}\\
            \huge{Groß}\\
            \Huge{Groß}
        \end{codeBlock}
        %\end{noindent}
    \end{frame}

    \begin{frame}[c, fragile]
        \slidehead{}
        %\begin{noindent}
        \begin{codeBlock}[]{
            minted language=latex,
            listing and text,
            sidebyside,
            righthand width=4.5cm,
            %center lower,
            fontlower=\setlength{\parskip}{\medskipamount}\setlength{\parindent}{0em},
            bicolor,
            colbacklower=\IfDarkModeTF{codebg!90}{codebg!20},
            title=\codeBlockTitle{Platz machen}
            }
            Zeile 1

            Zeile 2

            \vspace{1cm}
            Zeile 3\hspace{1cm}Hat Platz

            \vspace{-4mm}
            Zeile 4 ist zu nah an Zeile 3
        \end{codeBlock}
        %\end{noindent}
    \end{frame}

    \begin{frame}[c, fragile]
        \slidehead{}
        %\begin{noindent}
        \begin{codeBlock}[]{
            minted language=latex,
            listing and text,
            sidebyside,
            righthand width=4.5cm,
            %center lower,
            fontlower=\setlength{\parskip}{\medskipamount}\setlength{\parindent}{0em},
            bicolor,
            colbacklower=\IfDarkModeTF{codebg!90}{codebg!20},
            title=\codeBlockTitle{Farben}
            }
            \textcolor{red}{Rot}\\
            \textcolor{green}{Grün}\\
            \textcolor{blue!50!red}{Mischung}\\
            \textcolor{blue!50!red!50}{Mischung}
        \end{codeBlock}
        %\end{noindent}
    \end{frame}

    \subsection{Titelei}
    \begin{frame}[c, fragile]
        %\slidehead{}
        \begin{columns}[c]
            \begin{column}{.5\textwidth}
                \inputCode[]{
                    minted language=latex,
                    title=\codeBlockTitle{Titelei},
                }{\codeDir/sharelatex-template.tex}
            \end{column}%
            \begin{column}{.5\textwidth}
                \centering
                \colorbox{white}{%
                    \color{black}
                    \includegraphics[width=\textwidth,height=5cm,keepaspectratio]{\codeDir/sharelatex-template}
                }
            \end{column}
        \end{columns}
    \end{frame}
    \begin{frame}[c, fragile]
        %\slidehead{}
        \begin{columns}[c]
            \begin{column}{.5\textwidth}
                \inputCode[highlightlines={4-6}]{
                    minted language=latex,
                    title=\codeBlockTitle{Titelei},
                }{\codeDir/sharelatex-template.tex}
            \end{column}%
            \begin{column}{.5\textwidth}
                \centering
                \colorbox{white}{%
                    \color{black}
                    \includegraphics[width=\textwidth,height=5cm,keepaspectratio]{\codeDir/sharelatex-template}
                }
            \end{column}
        \end{columns}
    \end{frame}
    \begin{frame}[c, fragile]
        %\slidehead{}
        \begin{columns}[c]
            \begin{column}{.5\textwidth}
                \inputCode[highlightlines={10}]{
                    minted language=latex,
                    title=\codeBlockTitle{Titelei},
                }{\codeDir/sharelatex-template.tex}
            \end{column}%
            \begin{column}{.5\textwidth}
                \centering
                \colorbox{white}{%
                    \color{black}
                    \includegraphics[width=\textwidth,height=5cm,keepaspectratio]{\codeDir/sharelatex-template}
                }
            \end{column}
        \end{columns}
    \end{frame}
    \begin{frame}[c, fragile]
        %\slidehead{}
        \begin{columns}[c]
            \begin{column}{.5\textwidth}
                \inputCode[highlightlines={12}]{
                    minted language=latex,
                    title=\codeBlockTitle{Titelei},
                }{\codeDir/sharelatex-template.tex}
            \end{column}%
            \begin{column}{.5\textwidth}
                \centering
                \colorbox{white}{%
                    \color{black}
                    \includegraphics[width=\textwidth,height=5cm,keepaspectratio]{\codeDir/sharelatex-template}
                }
            \end{column}
        \end{columns}
    \end{frame}

    \subsection{Mit \LaTeX arbeiten}
    \urlslide[Sharelatex-Instanz der TU-Darmstadt(langsam, nicht empfohlen)]{https://sharelatex01.ca.hrz.tu-darmstadt.de}
    \urlslide[Coder-Instanz der Tudalgo(schneller, empfohlen falls nicht lokal installiert)]{https://coder.tudalgo.org}
    \begin{frame}[c,fragile]
        \slidehead{}
        \fatsf{Docker-Image}
        \verb+ghcr.io/tudalgo/algotex:latest+
        \begin{itemize}
            \item Makefile: \verb+make DOCKER=true+ (siehe Beispielprojekt)
            \item Vscode Devcontainer: \verb+Remote-Containers: Reopen in Container+ (siehe Beispielprojekt)
            \item manuell: \verb+docker run -it --rm -v $(pwd):/project -w /project ghcr.io/tudalgo/algotex:latest latexmk --shell-escape -synctex=1 -interaction=nonstopmode -file-line-error -lualatex folien.tex+ (ggf. mehr flags)
        \end{itemize}
    \end{frame}
    \begin{frame}[c,fragile]
        \slidehead{}
        \fatsf{Manuell}
        \begin{itemize}
            \item TeX-Live full installieren
            \item TU-Logo installieren
            \item AlgoTeX installieren
        \end{itemize}
    \end{frame}
    \subsection{Übungsphase}
    \begin{frame}[c]
        \slidehead{}
        \centering
        5 Minuten Übungsphase
        \begin{columns}[c]
            \begin{column}{.5\textwidth}
                \begin{figure}
                    \centering
                    \qrcode[height=3cm]{https://sharelatex01.ca.hrz.tu-darmstadt.de}
                    \caption{Coder-Instanz der Tudalgo\\\url{https://coder.tudalgo.org}}
                \end{figure}
            \end{column}%
            \begin{column}{.5\textwidth}
                \begin{figure}
                    \centering
                    \qrcode[height=3cm]{https://github.com/ophase-2022-23/LaTeX-Workshop/releases/latest}
                    \caption{Link zu Folien + Übungen\\\url{https://github.com/ophase-2022-23/LaTeX-Workshop/releases/latest}}
                \end{figure}
            \end{column}
        \end{columns}
    \end{frame}
\end{document}
