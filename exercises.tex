\documentclass[
    inlineshortcut=java, % Befehl \inlinejava{<code>} Konfigurieren
    corporatedesign, % TU-Design
    boxarc, % Abgerundete Ecken bei den Boxen
    %titleprefix={LaTeX-Workshop}, % FOP-Vorlage benutzen
    % dark_mode,
]{algoexercise}

%%------------%%
%%--Packages--%%
%%------------%%

% \usepackage{audutils}
\usepackage{fopbot}
\usepackage{menukeys}
\usepackage{booktabs}

%%---------------------------%%
%%--Dokumenteneinstellungen--%%
%%---------------------------%%

\duedate{28.10.2022 bis 23:50 Uhr}
\author{Ruben Deisenroth} % Übungsblattbetreuer
\subtitle{Übungsaufgaben}
\fachbereich{Informatik}
\semester{Sommersemester 2023} % z.B. SoSe 2022 oder WiSe 2022/2023
\sheetnumber{0} % Einstellige Nummern werden mit 0 aufgefüllt
\slides{01a} % Die Relevanten Foliensätze
\topics{\LaTeX-Basics, Formeln, Abbildungen} % Für das Übungsblatt relevante Themengebiete
\title[LaTeX-Workshop]{LaTeX-Workshop}
\version{1.0-SNAPSHOT}
\date{\today}
\graphicspath{{./pictures/}}

%%----------------------------%%
%%--Stilistische Anpassungen--%%
%%----------------------------%%

\ExplSyntaxOn
% Author
\renewcommand*{\author}[1]{
    \seq_gset_split:Nnn \g_ptxcd_author_seq {\and} {#1}
    \seq_if_empty:NF \g_ptxcd_author_seq {\tl_gset:Nn \printAuthor {\int_compare:nTF{\seq_count:N \g_ptxcd_author_seq >
                1}{Autoren}{Autor}:~\hfill\seq_use:Nnnn \g_ptxcd_author_seq {~\authorandname{}~} {,~} {~\authorandname{}~}\par}}
}
\termLeft{}
\termRight{}
\term{
    Author: \getAuthor{}\hfill{}\printDate{}\\
    \printTopics{}
    % \printSlides{}\\
    % \printDuedate{}
}
\ExplSyntaxOff

\ConfigureHeadline{
    headline={algo-min}
}

\IfDarkModeT{
    \colorlet{fopbot@lightgray}{black!80!gray!98!blue}
    \colorlet{fopbot@dark_gray}{white!20!gray!98!blue}
    \colorlet{fopbot@coin}{orange!90}
}

%%-------------------------%%
%%--Beginn des Dokumentes--%%
%%-------------------------%%

\begin{document}%

    %%-----------%%
    %%--Titelei--%%
    %%-----------%%

    \maketitle{}
    \vspace{-1em}
    \hue{Übungsaufgaben zum \LaTeX-Workshop}{Aller Anfang ist schwer...}{\getPointsTotal{}}

    %\tableofcontents

    %%------------------------------%%
    %%--Verbindliche Anforderungen--%%
    %%------------------------------%%

    %\UseSnippet{exercise-introduction-fop-2223}

    %%--------------%%
    %%--Einleitung--%%
    %%--------------%%
    % \section*{Einleitung}
    % Im Laufe des Semesters werden Sie jede Woche eine Programmierübung bearbeiten und abgeben, um die Studienleistung und ggf. den Klausurbonus zu erreichen. Insgesamt gibt es 13 Übungsblätter und ein FOP-Projekt. Die Übungsblätter sind Einzelabgaben, das FOP-Projekt ist eine Gruppenabgabe mit größerem Umfang und wird voraussichtlich in vierer-Gruppen Abzugeben sein. \href{https://moodle.informatik.tu-darmstadt.de/mod/page/view.php?id=49970#studienleistung}{Informationen zu Studienleistung und zum Klausurbonus finden Sie im Moodle-Kurs.}

    % Ende der Titelseite
    % \clearpage{}

    %%------------------------%%
    %%--Beginn der Hausübung--%%
    %%------------------------%%

    \begin{task}[points=auto]{\LaTeX-Basics}
        \begin{subtask*}[points=2]{Hello World}
            \begin{enumerate}
                \item Erstellen Sie ein minimales Textdokument mit Klasse \texttt{article} und \texttt{a4paper} mit dem Text \verb+Hello World+.
                \item Erstellen Sie eim Textdokument mit 2 seiten Blindtext.
            \end{enumerate}
        \end{subtask*}
        \vspace{-2em}
        \begin{subtask*}[points=1]{Textarten}
            Tippen Sie den folgenden Text ein:

            \begin{grayInfoBox}
                \Large Das \emph{Einzige}, \large was wir \enquote{\textbf{wissen}}, ist, \normalsize dass wir \textcolor{\IfDarkModeTF{cyan}{blue}}{nichts} \fatsf{wissen}. \small Und das ist \texttt{wichtig}.
            \end{grayInfoBox}
        \end{subtask*}
        \begin{subtask*}[points=1]{Zeilenumbrüche}
            Tippen sie eine Art \enquote{Tic-Tac-Toe}:

            \begin{grayInfoBox}[fontupper=\setlength{\parskip}{0cm}\setlength{\parindent}{0em}, center upper]
                \textcolor{red}{X}\hspace{1cm}O\hspace{1cm}X

                \vspace{1cm}O\hspace{1cm}\textcolor{red}{X}\hspace{1cm}O

                \vspace{1cm}X\hspace{1cm}O\hspace{1cm}\textcolor{red}{X}
            \end{grayInfoBox}
        \end{subtask*}
        \begin{subtask*}[points=1]{Trennhilfe}
            Bin gerade zu unkreativ um mir nen text auszudenken, also bischen selber basteln\^{}\^{}
        \end{subtask*}
    \end{task}
    \clearpage{}
    \begin{task}[points=auto]{Mathematische Formeln}
        \begin{subtask*}[points=3]{Binomische Formeln}
            Tippen Sie die folgenden binomischen Formeln ein:
            \begin{grayInfoBox}
                \begin{align}
                    (a+b)^2    & = a^2 + 2ab + b^2 \\
                    (a-b)^2    & = a^2 - 2ab + b^2 \\
                    (a+b)(a-b) & = a^2 - b^2
                \end{align}
            \end{grayInfoBox}
        \end{subtask*}
        \begin{subtask*}[points=1]{mitternachtsformel}
            Tippen Sie die folgende Formel ein:
            \begin{grayInfoBox}
                \fatsf{Mitternachtsformel}
                Für eine Polynom zweiten Grades in der Form $a\cdot x^2+b\cdot x + c$ gilt für $a,b,c \in \mathbb{R}$ immer:%

                \begin{align}
                    x_{1/2} & =\frac{-b\pm\sqrt{b^{2}-4\cdot a \cdot c}}{2\cdot a}
                \end{align}

            \end{grayInfoBox}
        \end{subtask*}
    \end{task}
\end{document}
