\documentclass[
    inlineshortcut=java, % Befehl \inlinejava{<code>} Konfigurieren
    corporatedesign, % TU-Design
    boxarc, % Abgerundete Ecken bei den Boxen
    %titleprefix={LaTeX-Workshop}, % FOP-Vorlage benutzen
    % dark_mode,
]{algoexercise}

%%------------%%
%%--Packages--%%
%%------------%%

% \usepackage{audutils}
\usepackage{FOPBot}
\usepackage{menukeys}
\usepackage{booktabs}
\usepackage{placeins}

\ExplSyntaxOn
\def\summerOrWinter{Winter}
\bool_if:nT {\int_compare_p:n {\the\month > 4} && \int_compare_p:n {\the\month < 10}}{
    \def\summerOrWinter{Sommer}
}
\def\ophase{\summerOrWinter{}ophase}
\ExplSyntaxOff

%%---------------------------%%
%%--Dokumenteneinstellungen--%%
%%---------------------------%%

\duedate{28.10.2022 bis 23:50 Uhr}
\subtitle{Übungsaufgaben}
\fachbereich{Informatik}
\dozent{Ruben Deisenroth}
\semester{\summerOrWinter{}semester \the\year} % z.B. SoSe 2022 oder WiSe 2022/2023
\sheetnumber{0} % Einstellige Nummern werden mit 0 aufgefüllt
\slides{01a} % Die Relevanten Foliensätze
\topics{\LaTeX-Basics, Formeln, Abbildungen} % Für das Übungsblatt relevante Themengebiete
\title[LaTeX-Workshop]{LaTeX-Workshop}
\version{1.0}
\date{\today}
\graphicspath{{./pictures/}}

%%----------------------------%%
%%--Stilistische Anpassungen--%%
%%----------------------------%%

\ExplSyntaxOn
\termLeft{}
\termRight{}
\term{
    Author:~\fatsf{\getAuthor{}}\hfill{}\printDate{}\\
    \printTopics{}
    % \printSlides{}\\
    % \printDuedate{}
}
\ExplSyntaxOff

\author{Ruben Deisenroth} % Übungsblattbetreuer

\ConfigureHeadline{
    headline={algo-min}
}

\IfDarkModeT{
    \colorlet{fopbot@lightgray}{black!80!gray!98!blue}
    \colorlet{fopbot@dark_gray}{white!20!gray!98!blue}
    \colorlet{fopbot@coin}{orange!90}
}

%%-------------------------%%
%%--Beginn des Dokumentes--%%
%%-------------------------%%

\begin{document}%

    %%-----------%%
    %%--Titelei--%%
    %%-----------%%

    \maketitle{}
    \vspace{-1em}
    \hue{Übungsaufgaben zum \LaTeX-Workshop}{Aller Anfang ist schwer...}{\getPointsTotal{}}

    %\tableofcontents

    %%------------------------------%%
    %%--Verbindliche Anforderungen--%%
    %%------------------------------%%

    %\UseSnippet{exercise-introduction-fop-2223}

    %%--------------%%
    %%--Einleitung--%%
    %%--------------%%
    % \section*{Einleitung}

    % Ende der Titelseite
    % \clearpage{}

    %%------------------------%%
    %%--Beginn der Hausübung--%%
    %%------------------------%%

    \begin{task}[points=auto]{\LaTeX-Basics}
        \begin{subtask*}[points=2]{Hello World}
            \begin{enumerate}
                \item Erstellen Sie ein minimales Textdokument mit Klasse \texttt{article} und \texttt{a4paper} mit dem Text \verb+Hello World+.
                \item Erstellen Sie eim Textdokument mit 2 seiten Blindtext.
            \end{enumerate}
        \end{subtask*}
        \vspace{-2em}
        \begin{subtask*}[points=1]{Textarten}
            Tippen Sie den folgenden Text ein:

            \begin{grayInfoBox}
                \Large Das \emph{Einzige}, \large was wir \enquote{\textbf{wissen}}, ist, \normalsize dass wir \textcolor{\IfDarkModeTF{cyan}{blue}}{nichts} \fatsf{wissen}. \small Und das ist \texttt{wichtig}.
            \end{grayInfoBox}
        \end{subtask*}
        \begin{subtask*}[points=1]{Zeilenumbrüche}
            Tippen Sie eine Art \enquote{Tic-Tac-Toe}:

            \begin{grayInfoBox}[fontupper=\setlength{\parskip}{0cm}\setlength{\parindent}{0em}, center upper]
                \textcolor{red}{X}\hspace{1cm}O\hspace{1cm}X

                \vspace{1cm}O\hspace{1cm}\textcolor{red}{X}\hspace{1cm}O

                \vspace{1cm}X\hspace{1cm}O\hspace{1cm}\textcolor{red}{X}
            \end{grayInfoBox}

            \begin{hinweise}
                \begin{itemize}
                    \item Die Box ist nur zur Veranschaulichung da. Sie müssen nur den Text tippen.
                    \item Sie können \verb+\vspace{<länge>}+ und \verb+\hspace{<länge>}+ nutzen, um die Abstände zu setzen.
                \end{itemize}
            \end{hinweise}
        \end{subtask*}
        \clearpage{}
        \begin{subtask*}[points=1]{Trennhilfe}
            % Hier bitte selbst ein Beispiel ausdenken. Es kann ja nicht angehen, dass dieser Text einfach ungetrennt bleibt. Da\-bei könnt ihr das doch selbstverständlich!\^{}\^{}
            Fügen Sie für die folgenden Wörter Trennhilfen ein:
            \begin{itemize}
                \item Rindfleischetikettierungsüberwachungsaufgabenübertragungsgesetz
                \item Grundstücksverkehrsgenehmigungszuständigkeitsübertragungsverordnung
            \end{itemize}
            Das sieht dann in etwa so aus:
            \begin{figure}[ht!]
                \centering
                \begin{subfigure}[t]{.5\textwidth}
                    \centering
                    \begin{tcolorbox}[width=3cm]
                        Rind\-flei\-sche\-ti\-ket\-tie\-rungs\-über\-wa\-chungs\-auf\-ga\-ben\-über\-tra\-gungs\-ge\-setz
                    \end{tcolorbox}
                \end{subfigure}%
                \begin{subfigure}[t]{.5\textwidth}
                    \centering
                    \begin{tcolorbox}[width=3cm]
                        Grund\-stücks\-ver\-kehrs\-ge\-neh\-mi\-gungs\-zu\-stän\-dig\-keits\-über\-tra\-gungs\-ver\-ord\-nung
                    \end{tcolorbox}
                \end{subfigure}
                \caption{Beispiel für Trennhilfen, jeweils mit 3cm breiter Tcolorbox dargestellt}
                \label{fig:example-hyphenation}
            \end{figure}
            \FloatBarrier
            Sie können ihr Ergebnis ebenfalls in einer Tcolorbox testen:
            \begin{codeBlock}[
                % highlightlines={7}
                escapeinside=||,
            ]{
                minted language=latex,
                listing and text,
                sidebyside,
                %righthand width=4.1cm,
                %center lower,
                bicolor,
                colbacklower=\IfDarkModeTF{codebg!90}{codebg!20},
                title=\codeBlockTitle{tcolorbox Beispiel},
                top=0pt,
                bottom=0pt,
            }
            % needs package tcolorbox
            \begin{tcolorbox}[width=3cm]
                <Ihr Text>
            \end{tcolorbox}
        \end{codeBlock}
        \end{subtask*}
    \end{task}
    % \clearpage{}
    \begin{task}[points=auto]{Mathematische Formeln}
        \begin{subtask*}[points=3]{Binomische Formeln}
            Tippen Sie die folgenden binomischen Formeln ein:
            \begin{grayInfoBox}
                \begin{align}
                    (a+b)^2    & = a^2 + 2ab + b^2 \\
                    (a-b)^2    & = a^2 - 2ab + b^2 \\
                    (a+b)(a-b) & = a^2 - b^2
                \end{align}
            \end{grayInfoBox}
        \end{subtask*}
        \begin{subtask*}[points=1]{mitternachtsformel}
            Tippen Sie die folgende Formel ein:
            \begin{grayInfoBox}
                \fatsf{Mitternachtsformel}
                Für eine Polynom zweiten Grades in der Form $a\cdot x^2+b\cdot x + c$ gilt für $a,b,c \in \mathbb{R}$ immer:%

                \begin{align}
                    x_{1/2} & =\frac{-b\pm\sqrt{b^{2}-4\cdot a \cdot c}}{2\cdot a}
                \end{align}

            \end{grayInfoBox}
        \end{subtask*}
    \end{task}
\end{document}
